% Dokumentklasse; Optionen in eckige klammern
\documentclass{scrartcl}
% Ein paar Pakete für Komfort, etwa Umlaute
\usepackage{german}
\usepackage[utf8]{inputenc}

% zum Einbinden von Grafiken
\usepackage{graphicx}

% bessere kontrolle über figures
\usepackage{float}

% für mehr Mathe-Optionen (AMS=American MAth Society)
\usepackage{amssymb}
\usepackage{amsmath}

% zum Einbetten von Code
\usepackage{listings}

% Anpassen von Kopf- und Fußzeilen.
% weitere Optionen möglich, etwa footsepline (mehr Informationen in der KOMA-Skript Dokumentation)
\usepackage[headsepline,automark]{scrlayer-scrpage}
\pagestyle{scrheadings}
\automark[section]{section}

% äußere Kopfzeile (bei einseitigen Dokumenten rechts)
\ohead[]{Alexander Hornig \\ Till}
% innere Kopfzeile (bei einseitigen Dokumenten links)
\ihead{\headmark}
% mittlere Kopfzeile
\chead{Stochastik - Blatt 3}
% Fußzeile analog
\ofoot[]{}
\ifoot[]{}
\cfoot[]{}

\begin{document}


%%%%%%%%%
\section*{Aufgabe 1}
  \subsection*{a)}

  \subsection*{b)}


%%%%%%%%%
\section*{Aufgabe 2}
  \subsection*{a)}

  \subsection*{b)}


%%%%%%%%%
\section*{Aufgabe 3}
	Die Menge der möglichen Antworten ist sehr groß, daher \\
	$\Rightarrow$ Ziehen ihne Zurücklegen $\approx$ Ziehen mit Zurücklegen \\
	$\Rightarrow$ Binomialverteilung \\
	\\
	$p_k = {n \choose k} p^k (1-p)^{n-k}$ \\
	$n = 7$
  \subsection*{a)}
	$\mathbb{P}(\{\text{7 korrekte Antworten}\}) = {7 \choose 7} 0,7^7 (1-0,7)^{7-7} = 0,7^7 = 0,0823543$
  \subsection*{b)}
	$\mathbb{P}(\{\text{0 korrekte Antworten}\}) = {7 \choose 0} 0,7^0 (1-0,7)^{7-0} = 0,3^7 = 0,0002187$
  \subsection*{c)}
	$\mathbb{P}(\{\text{mindestens 1 korrekte Antwort}\}) = 1 - \mathbb{P}(\{\text{0 korrekte Antworten}\}) \stackrel{b)}{=} 1 - 0,3^7 = 0,9997813$
  \subsection*{d)}
	$\mathbb{P}(\{\text{3 korrekte Antworten}\}) = {7 \choose 3} 0,7^3 (1-0,7)^{7-3} = 35 * 0,7^3 * 0,3^4 = 0,0972405$

%%%%%%%%%
\section*{Aufgabe 4}
	\begin{flalign*}
						& P(A \cap B) = P(A) * P(B)  &|:P(B) && \\
		\Leftrightarrow	& \frac{P(A \cap B)}{P(B)} = P(A) && \\
		\stackrel{\text{Def 6.1}}{\Leftrightarrow} & P(A|B) = P(A) && \\
		\Leftrightarrow	& P(A|B) = \frac{P(A) * P(B^c)}{P(B^c)} && \\
		\Leftrightarrow	& P(A|B) = P(A|B^c) && \\
		&& \square
	\end{flalign*}


%%%%%%%%%
\section*{Aufgabe 5}
	zwei Möglichkeiten
	\subsection*{[1]}
		Die größere Zahl wird als erste gezogen: \\
		Es gibt 21 Zahlen $\geq 80$ und insgesamt 100 Zahlen: \\
		$\Rightarrow \mathbb{P}([1]) = \frac{21}{100}$

	\subsection*{[2]}
		Die größere Zahl wird als zweite gezogen: \\
		Es gibt immer noch 21 Zahlen $\geq 80$, da die erste $\leq 20$ ist und insgesamt git es noch 99 Zahlen: \\
		$\Rightarrow \mathbb{P}([2]) = \frac{21}{99}$

	\subsection*{Gesamt}
		$\mathbb{P}(\{\text{größere Zahl $\geq$ 80}\} | \{\text{kleinere Zahl $\leq$ 20}\}) = \frac{21}{100} * \frac{21}{99} \approx 0,0445$

\end{document}
